\documentclass[11pt]{article}
\usepackage[utf8]{inputenc}
\usepackage{amsmath}
\usepackage{xcolor}
\usepackage[a4paper, total={6in, 8in}]{geometry}

\title{\vspace{-4.0cm}Sweeping Preconditioner with Moving Perfectly Matched Layer for the Helmholtz Equation}
\author{Alex Bocchieri}
\date{}


\begin{document}
\maketitle

\section{Introduction}
The Helmholtz equation defined as $\Delta u(x) + \frac{\omega^2}{c^2(x)}u(x)=f(x)$ 
is a time-independent factorization of the wave equation, where $w$ is angular 
frequency, $c(x)$ is the velocity field, and $f(x)$ is the external force. 
Many different approaches exist for numerically solving the Helmholtz equation 
in a discretized system. 

Boundary conditions must be handled properly when numerically solving a 
wave equation on a finite grid. The Sommerfeld condition for waves that propagate to an 
infinite distance is not satisfied when computing on a finite domain. This leads to 
non-physical reflections at the boundaries \cite{erlangga2008advances}. The typical 
correction is to impose absorbing boundary conditions. The perfectly matched layer (PML) 
\cite{berenger1994perfectly} is one common method for absorbing waves at the boundaries. 
The PML method adds a damping region around the original domain (or within it) and 
accordingly modifies the equation we are solving (Helmholtz). The PML reduces non-physical
reflections and provides a more favorable system to solve for, but it also adds
computational cost.

Certain iterative methods can be applied for solving the Helmholtz equation. They require 
preconditioning due to the Helmholtz system's indefiniteness. The methods in 
\cite{engquist2011pml} and \cite{engquist2011matrix} apply a sweeping preconditioner and
iteratively solve the system using GMRES \cite{saad1986gmres}. Both methods use the PML to arrive
at initial system. Their key observation is that the system can be factorized into a form
where \textcolor{red}{block matrices correspond to Green's function and are highly compressible.}
In \cite{engquist2011matrix}, these matrices are approximated using the hierarchical matrices. 
In \cite{engquist2011pml}, these matrices are represented by inverting a Helmholtz system with 
a moving PML. Both methods take $O(N)$ time, thereby improving on direct methods described in 
the following paragraph.

The frontal solver \cite{irons1970frontal} is a direct method for solving sparse 
systems in a manner similar to Gaussian elimination. The frontal solver extracts 
small blocks (fronts) within the sparse $A$ matrix for eliminating variables/equations. 
The multifrontal solver \cite{duff1983multifrontal} extends this method to process 
multiple independent fronts in parallel. For a 2D Helmholtz problem with $N=n^2$ unknowns, 
the multifrontal method takes $O(N^{3/2})$ time and $O(N \log N)$ space. The method 
becomes inefficient for 3D problems with $N=n^3$ unknowns, where it takes $O(N^2)$ 
time and $O(N^{4/3})$ space \cite{engquist2011matrix}. Direct methods also suffer 
from fill-in \cite{erlangga2008advances} that occurs during the Gaussian elimination process.





\bibliographystyle{plain}
\bibliography{bibliography}


\end{document}

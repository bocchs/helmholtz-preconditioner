\documentclass[11pt]{article}
\usepackage[utf8]{inputenc}
\usepackage{amsmath}
\usepackage[a4paper, total={6in, 8in}]{geometry}

\title{\vspace{-4.0cm}Sweeping Preconditioner with Moving Perfectly Matched Layer for the Helmholtz Equation}
\author{Alex Bocchieri}
\date{}


\begin{document}
\maketitle

\section{Introduction}
The Helmholtz equation defined as $\Delta u(x) + \frac{\omega^2}{c^2(x)}u(x)=f(x)$ is a time-independent factorization of the wave equation, where $w$ is angular frequency, $c(x)$ is the velocity field, and $f(x)$ is the external force. Many different approaches exist for numerically solving the Helmholtz equation in a discretized system. 

The frontal solver \cite{irons1970frontal} is a direct method for solving sparse systems in a manner similar to Gaussian elimination. The frontal solver extracts small blocks (fronts) within the sparse $A$ matrix for eliminating variables/equations. The multifrontal solver \cite{duff1983multifrontal} extends this method to process multiple independent fronts in parallel. For a 2D problem with $N=n^2$ unknowns, the multifrontal method takes $O(N^{3/2})$ time and $O(N \log N)$ space. The method becomes inefficient for 3D problems with $N=n^3$ unknowns, where it takes $O(N^2)$ time and $O(N^{4/3})$ space \cite{engquist2011matrix}. Direct methods also suffer from fill-in \cite{erlangga2008advances} that occurs during the Gaussian elimination process.

Iterative methods are an alternative approach that have a small memory requirement. They require preconditioning due to the Helmholtz system's indefiniteness. 



\cite{engquist2011pml} \cite{engquist2011matrix}




\bibliographystyle{plain}
\bibliography{bibliography}


\end{document}
